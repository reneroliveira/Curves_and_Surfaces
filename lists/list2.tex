\documentclass[12pt,letterpaper]{article}

\usepackage[brazilian]{babel}
\usepackage[utf8]{inputenc}
\usepackage[T1]{fontenc}

\usepackage{fullpage}
\usepackage{cancel}
\usepackage[top=2cm, bottom=4.5cm, left=2.5cm, right=2.5cm]{geometry}
\usepackage{amsmath,amsthm,amsfonts,amssymb,amscd}
\usepackage{lastpage}
\usepackage{enumerate}
\usepackage{fancyhdr}
\usepackage{mathrsfs}
\usepackage{xcolor}
\usepackage{graphicx}
\usepackage{listings}
\usepackage{hyperref}

\hypersetup{%
  colorlinks=true,
  linkcolor=blue,
  linkbordercolor={0 0 1}
}

\setlength{\parindent}{0.0in}
\setlength{\parskip}{0.05in}

% Edit these as appropriate
\newcommand\course{Rener Oliveira}
\newcommand\lcur{\mathcal{L}}
\newcommand{\linesep}{{\color{black} \rule{\linewidth}{0.5mm} }}
\newcommand{\rpos}{\mathbb{R}_{>0}}
\newcommand{\ex}[1]{\textcolor{blue}{\textbf{Exercício #1}}}
\newcommand{\sol}[1]{\textbf{Solução #1}}
\newcommand{\blue}[1]{{\color{blue}{#1}}}
\pagestyle{fancyplain}
\headheight 35pt              % <-- Comment this line out for problem sets (make sure you are person #1)
\chead{\textbf{\Large Lista 2 \\ Curvas e Superfícies}}
\rhead{\small{\course \\ \today}}
\lfoot{}
\cfoot{}
\rfoot{\small\thepage}
\headsep 1.5em

\begin{document}
	
	\begin{enumerate}
		
		\item [\ex{1}] \blue{Desenhe em ambiente computacional, utilizando sistemas de computação simbólica, incluindo a animação do vetor tangente percorrendo as seguintes parametrizações da parábola $\alpha(t)=(t,t^2)$ e $\gamma(t) = (t^3,t^6)$. Mostre que $\alpha$ é curva regular e $\gamma$ não é regular. Qual seria a função naturalmente candidata a ser uma reparametrização entre as duas parametrizações? Porque falha?
	}
		
		\item [\sol{1}] Veja que $\alpha'(t)=(1,2t^2)$ e $\gamma'(t)=(3t^2,6t^5)$. Por conta da coordenada constante, $\alpha'(t)\neq 0 ~~\forall ~t\in\mathbb{R}$, logo é regular. Já $\gamma'$ é anulada para $t=0$, logo não é regular.
		
		Antes de definir reparametrização vamos fixar algumas coisas. Seja $I_1$ o domínio de $\alpha$ e $I_2$ o domínio de $\gamma$; Vamos considerar neste caso $I_1=I_2=\mathbb{R}$. $\gamma$ é reparametrização de $\alpha$ se existe um \textbf{difeomorfismo} $\phi:I_2\to I_1$ tal que $\gamma(t)=\alpha(\phi(t))~\forall t \in I_2$.
		
		Assim a função naturalmente candidata à reparametrização de $\alpha$ é $\phi(t)=t^3$.
		
		Porque falha? Pois $\phi$ não é um difeomorfismo, logo não é reparametrização segundo a definição dada! Sua inversa $\phi^{-1}(t)=t^{1/3}$ não é diferenciável em todos os pontos do domínio, pois a derivada $\frac13t^{-2/3}$ não está definita em $t=0$.
		
		
		\linesep
		\item [\ex{2}] \blue{Mostre que as curvas regulares $\alpha(t)=(t,e^t)$, $t\in\mathbb{R}$ e $\beta(s) = (\log(s),s)$, $s\in(0,\infty)$ têm o mesmo traço.
		}
	\item [\sol{2}] Queremos provar que $\alpha(\mathbb{R})=\beta(\rpos)$. Tome 
	
	\begin{align*}
	\phi:\rpos&\to\mathbb{R}\\
	\phi(s)&=\log(s)
	\end{align*}
	
	Afirmo que $\phi$ é bijetiva. A injetividade segue da monotonicidade estrita do logaritmo\footnote{Uma outra prova (rascunho) rápida: $\log(x)=\log(y)=b\Rightarrow e^b=x\text{ e }e^b=y\Rightarrow x=y$}. Segue um rascunho da prova da sobrejetividade:
	
	Usando a definição $\log x:=\displaystyle\int_1^x\frac1udu$, pelo Teorema Fundamental do Cálculo\cite{lima1981curso}, temos que $\log$ é diferenciável em todo seu domínio ($\rpos$), disso segue a continuidade. 
	
	Para todo inteiro $k$ no contradomínio $\mathbb{R}$ conseguimos encontrar um elemento $x$ no domínio tal que $\log(x)=k$ usando que $\log(e)=1$ e a propriedade $\log(ab)=\log(a)+\log(b)$, os inteiros positivos são atingidos pelas potências positivas de $e$ e os negativos pelas potências negativas. Pela continuidade, e pelo Teorema do Valor Intermediário\cite{lima1981curso}, segue que $\forall k\in\mathbb{Z}$, vale que $\forall r \in (k,k+1),~\exists x\in\rpos$ tal que $\log(x)=r$, que prova a sobrejetividade.
	
	Acima já demos o motivo de $\phi$ ser diferenciável, se verificarmos que $\phi^{-1}$ é diferenciável, teremos que $\phi$ é um difeomorfismo. De fato $\phi^{-1}:\mathbb{R}\to\rpos$ definida pel exponencial, não só é diferenciável como pertence a $C^{\infty}$.
	
	Sendo assim $\phi$ é difeomorfismo. Note que $\beta(s)=\alpha(\phi(s))$.
	
	O traço de $\alpha$, $\alpha(\mathbb{R})$ é o conjunto $\{(x,y);x\in \mathbb{R},~y\in\rpos\}$, pois a segunda entrada exponencial é estritamente positiva.
	
	O difeomorfismo $\phi$ (por ser bijetivo) leva todos os pontos $s\in\rpos$ em $\mathbb{R}$. Sendo assim,
	
	 $$\beta(\rpos)=\alpha(\phi(\rpos))=\alpha(\mathbb{R})$$
	 
	 \textbf{Como queríamos demonstrar.}
	
	
	\linesep
		\item [\ex{3}] \blue{Calcule o comprimento de arco das seguintes curvas:}
			
			\begin{enumerate}[a.]
				\color{blue}
				\item $\alpha(t)=(3\cosh 2t,3\sinh 2t,6t), ~~t\in[0,\pi]$
				\item Catenária: $\gamma(t)=(t,\cosh(t))$, a partir do ponto $(0,1)$.
			\end{enumerate}
		
		
		
		
		\item [\sol{3}]
		Usaremos a fórmuma $\displaystyle\int_{t_0}^{t}||\gamma'(u)||du$ para calcular o comprimento de arco de $\gamma$ entre $t_0$ e $t$.
		\begin{enumerate}[a.]
			\item Para $\alpha(t)=(3\cosh 2t,3\sinh 2t,6t)$ com $t\in[0,\pi]$, temos
			\begin{align*}
				&\int_{0}^{\pi}||\alpha'(u)||du\\
				=&\int_{0}^{\pi}||(3\cosh 2u,3\sinh 2u,6u)'||du\\
				=&\int_{0}^{\pi}||(6\sinh 2u,6\cosh2u,6)||du\\
				=&\int_{0}^{\pi}\left(36(\sinh^22u+\cosh^22u+1)\right)^{1/2}du\\
				=&\int_{0}^{\pi}6(2\cosh^22u-1+1)^{1/2}du\\
				=&6\sqrt2\int_{0}^{\pi}\cosh2u~du\\
				=&3\sqrt2\left.\sinh2u\right|_ {0}^{\pi}\\
				=&3\sqrt2(\sinh(2\pi)-\cancelto{0}{\sinh(0)})\\
				=&\frac32\sqrt2(e^{2\pi}-e^{-2\pi})
			\end{align*}
		
		\item Para $\gamma(t)=(t,\cosh(t))$, a partir de $(0,1)$, quer dizer a partir de $t=1$, e iremos calcular a distância parametrizada por $s$.
		
		\begin{align*}
			&\int_{1}^{s}||(t,\cosh(t))'||dt\\
			=&\int_{1}^{s}||(1,\sinh(t))||dt\\
			=&\int_{1}^{s}\left(1+\sinh^2t\right)^{1/2}dt\\
			=&\int_{1}^{s}(\cosh^2t)^{1/2}dt\\
			=&\int_{1}^{s}\cosh t ~dt\\
			=&\left.\sinh t\right|_{t=1}^{t=s}\\
			=&\sinh(s)-\cancelto{0}{\sinh(0)}\\
			=&\sinh(s)
		\end{align*}
		\end{enumerate}
		
		\linesep
		
		\item[\ex{4}] \blue{\textit{Mudanças de parâmetro:}}
		
		\begin{enumerate}[a.]
			\color{blue}
			\item Demonstrar que $s(\theta)=\dfrac{\theta^2}{\theta^2+1}$ é uma mudança de parâmetro diferenciável que transforma o invervalo $(0,\infty)$ no intervalo $(0,1)$.
			
			
			
			\item Mostrar que a função $\lambda:(-1,1)\to(-\infty,+\infty)$ definida por $\lambda(t):=\tan(\pi t/2)$ é uma mudança de parâmetro.
			
			\item Provar que qualquer curva pode ser reparametrizada de forma tal que o domínio da reparametrização seja um intervalo de extremos $0$ e $1$.
		\end{enumerate}
	
	\item [\sol{4}] 
	\begin{enumerate}[\textbf{a.}]
		\item Primeiramente precisamos mostrar que $s(\rpos)=(0,1)$, ou seja, que $s(\rpos)\subseteq (0,1)$ e $(0,1)\subseteq s(\rpos)$
		
		Tome $x\in s(\rpos)$, é fácil ver que $x>1$, pois $x$ é divisão de dois números estritamente positivos: $\dfrac{\theta^2}{\theta^2+1},\theta\in\rpos$, para provar a continência em $(0,1)$ basta provar que $x<1$. Veja que, dado $\theta>0$, temos $\dfrac{\theta^2}{\theta^2+1}=\dfrac{1}{1+1/\theta^2}<1$, ou seja, $x<1$, o que implica finalmente, que $x\in(0,1)$.
		
		Para provar a continência inversa, tome $x\in(0,1)$, vamos provar que $x\in s(\rpos)$ exibindo um $\theta\in\rpos$ tal que $s(\theta)=x$. Com um algebrismo simples\footnote{Não houve nenhuma divisão por zero no processo, pode confiar}, chega-se em $\theta=\left(\dfrac{x}{1-x}\right)^{1/2}\in\rpos$, veja então, que $s(\theta)=\dfrac{1}{1+1/\theta^2}=\dfrac{x}{1+(1-x)/x}=\dfrac{x^2}{x+1-x}\cdot\dfrac1x=x^2/x=x$
		
		Logo $(0,1)\subseteq s(\rpos)$, e portanto $(0,1)= s(\rpos)$.
		
		Agora resta provar que a aplicação é um difeomorfismo. A sobrejetividade já for demonstrada do fato de que $(0,1)\subseteq s(\rpos)$ que equivale à $\forall x\in(0,1),\exists \theta\in\rpos$ tal que $s(\theta)=x$.
		
		Para provar a injetividade, vamos pela contrapositiva da definição. $s(\theta_1)=s(\theta_2)\Rightarrow\theta_1=\theta_2$:
		
		\begin{align*}
			&~s(\theta_1)=s(\theta_2)\\
			\Rightarrow&~\dfrac{1}{1+1/\theta_1^2}=\dfrac{1}{1+1/\theta_2^2}\\
			\Rightarrow&~1+1/\theta_1^2=1+1/\theta_2^2\\
			\Rightarrow&~1/\theta_1^2=1/\theta_2^2\\
			\Rightarrow&~\theta_1^2=\theta_2^2
		\end{align*}
	
	Como $\theta_1,\theta_2>0$, então segue que $\theta_1=\theta_2$
	
	Conclui-se então que $s$ é bijetiva, e sua inversa $s^{-1}:(0,1)\to\rpos$, é dada por $s^{-1}(x)=\left(\dfrac{x}{1-x}\right)^{1/2}$.
	
	Vamos provar, por fim a diferenciabilidade de $s$ e $s^{-1}$.
	
	Tome as funções:
	
	\begin{align*}
		s_1:\rpos&\to\rpos\\
		x&\mapsto1/x^2\\
		s_2:\rpos&\to\mathbb{R}_{>1}\\
		x&\mapsto x+1	\\
		s_3:\mathbb{R}_{>1}&\to(0,1)\\
		x&\mapsto1/x\\	
	\end{align*}
		
		Veja que $s(\theta)=(s_3\circ s_2 \circ s_1)(\theta)$ Sob seus domínios, todas as funções acima são diferenciáveis, omitirei a prova. Pela Regra da Cadeia\cite{lima1981curso}, a composição anterior é diferenciável $\forall \theta \in \rpos$. Podemos calcular a derivada usando regras usuais do cálculo na fórmula principal e chegamos em $s'(\theta)=\dfrac{2\theta}{(\theta^2+1)^2}$
		
		Pelo Corolário da Regra da Cadeia\cite{lima1981curso} que estabelece a derivada da função inversa, a função $s$ satisfaz todas as condições que implicam a diferenciabilidade de $s^{-1}$: S é diferenciável no seu domínio, $s^{-1}$ é contínua \footnote{O único ponto (real) de descontinuidade da expressão de $s^{-1}(x)=\left(\dfrac{x}{1-x}\right)^{1/2}$ seria $x=1$ que não está no domínio $(0,1)$} em todos os pontos da imagem de $s$, e $s'(\theta)>0,~\forall\theta\in\rpos$. Segue que $s^{-1}$ é diferenciável de derivada $1/s'$.
		
		Provamos então que $s$ é \textbf{bijetiva}, \textbf{diferenciável} e \textbf{sua inversa é diferenciável}, provando então que se trata de uma mudança de variáveis difeomorfa. $\blacksquare$
		
		\item Vamos usar a interpretação do termo "mudança de parâmetro" como sendo um homeomorfismo.
		
		Temos que provar então que $\lambda$ é bijetiva e contínua com inversa contínua.
		
		\textbf{(Injetividade)} Tome $t_1,t_2\in(-1,+1)$ arbitrários. Temos então,
		
		\begin{align}
			&\lambda(t_1)=\lambda(t_2)\nonumber\\
			\Rightarrow&~\tan(\pi t_1/2)=\tan(\pi t_2/2)\nonumber\\
			\Rightarrow&~\pi t_1/2=\pi t_ 2/2\label{tan}\\
			\Rightarrow&~ t_1 = t_2,\nonumber
		\end{align}
		
		onde \ref{tan} segue da bijetividade da tangente em $\left(-\dfrac{\pi}2,\dfrac{\pi}2\right)$. Segue então que $\lambda$ é injetiva em $(-1,1)$.
		
		\textbf{(Sobrejetividade)} Segue da bijetividade da tangente no mapeamento $\left(-\dfrac{\pi}2,\dfrac{\pi}2\right)\to\left(-\infty,\infty\right)$, que $\forall x \in \mathbb{R}$ existe $\theta\in\left(-\dfrac{\pi}2,\dfrac{\pi}2\right)$ tal que $\tan(\theta)=x$, sendo assim, é direto que $\exists~ t\in(-1,1)$ tal que $\tan(\pi t/2)=x$.
		
		\textbf{(Continuidade)} A continuidade de $\lambda$ segue da continuidade da tangente. Tome $\lambda^{-1}:\mathbb{R}\to(-1,1)$ dada por $\lambda^{-1}(x)=\dfrac{2\arctan(x)}{\pi}$.
		
		Segue da continuidade da arco tangente que $\lambda^{-1}$ é contínua em todos os reais, pois é apenas uma multiplicação por constante.
	\end{enumerate}
	
	\linesep
	
	\item[\ex{5}]\blue{
	Provar que a curva

$$\gamma(t)=\left(2t,\dfrac{2}{1+t^2}\right)$$

com $t>0$ é regular e é uma reparametrização de

$$\alpha(t)=\left(\dfrac{2\cos t}{1+\sin t},1+\sin t\right),~~t\in(-\pi/2,\pi/2)$$
}

	\item [\sol{5}]...
	
	\linesep
	
	\item[\ex{6}] \blue{Seja $\alpha(t)=\left(\frac{1}{\sqrt3}\cos t+\frac{1}{\sqrt2}\sin t,\dfrac{1}{\sqrt3}\cos t-\dfrac{1}{\sqrt2}\sin t\right)$. Reparametrizar $\alpha$ pelo comprimento do arco.
	}
	\item[\sol{6}] ...
	
	\linesep
	
	\item [\ex{7}] \blue{Mostre que, se todas as retas tangentes a uma curva regular passam por um mesmo ponto $P\in\mathbb{R}^2$ , então seu traço está contido em uma reta.}
	
	\item[\sol{7}]...
	
	\linesep
	
	\item [\ex{8}] \blue{Mostre que, se todas as retas normais a uma curva regular passam por um mesmo ponto $P\in\mathbb{R}^2$, então seu traço está contido em um círculo.}
	
	\item[\sol{8}]
	\end{enumerate}
	
\newpage

% \addcontentsline{toc}{section}{Referências}
\bibliographystyle{plain}
\bibliography{refs}
\end{document}