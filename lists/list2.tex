\documentclass[12pt,letterpaper]{article}

\usepackage[brazilian]{babel}
\usepackage[utf8]{inputenc}
\usepackage[T1]{fontenc}

\usepackage{fullpage}
\usepackage[top=2cm, bottom=4.5cm, left=2.5cm, right=2.5cm]{geometry}
\usepackage{amsmath,amsthm,amsfonts,amssymb,amscd}
\usepackage{lastpage}
\usepackage{enumerate}
\usepackage{fancyhdr}
\usepackage{mathrsfs}
\usepackage{xcolor}
\usepackage{graphicx}
\usepackage{listings}
\usepackage{hyperref}

\hypersetup{%
  colorlinks=true,
  linkcolor=blue,
  linkbordercolor={0 0 1}
}

\setlength{\parindent}{0.0in}
\setlength{\parskip}{0.05in}

% Edit these as appropriate
\newcommand\course{Rener Oliveira}

\newcommand{\linesep}{{\color{black} \rule{\linewidth}{0.5mm} }}

\newcommand{\ex}[1]{\textcolor{blue}{\textbf{Exercício #1}}}
\newcommand{\sol}[1]{\textbf{Solução #1}}
\newcommand{\blue}[1]{{\color{blue}{#1}}}
\pagestyle{fancyplain}
\headheight 35pt              % <-- Comment this line out for problem sets (make sure you are person #1)
\chead{\textbf{\Large Lista 2 \\ Curvas e Superfícies}}
\rhead{\small{\course \\ \today}}
\lfoot{}
\cfoot{}
\rfoot{\small\thepage}
\headsep 1.5em

\begin{document}
	
	\begin{enumerate}
		
		\item [\ex{1}] \blue{Desenhe em ambiente computacional, utilizando sistemas de computação simbólica, incluindo a animação do vetor tangente percorrendo as seguintes parametrizações da parábola $\alpha(t)=(t,t^2)$ e $\gamma(t) = (t^3,t^6)$. Mostre que $\alpha$ é curva regular e $\gamma$ não é regular. Qual seria a função naturalmente candidata a ser uma reparametrização entre as duasparametrizações? Porque falha?
	}
		
		\item [\sol{1}] ...
		
		
		\linesep
		\item [\ex{2}] \blue{Mostre que as curvas regulares $\alpha(t)=(t,e^t)$, $t\in\mathbb{R}$ e $\beta(s) = (\log(s),s)$, $s\in(0,\infty)$ têm o mesmo traço.
		}
	\item [\sol{2}] ...
	
	\linesep
		\item [\ex{3}] \blue{Calcule o comprimento de arco das seguintes curvas:}
			
			\begin{enumerate}[a.]
				\color{blue}
				\item $\alpha(t)=(3\cosh 2t,3\sinh 2t,6t), ~~t\in[0,\pi]$
				\item Catenária: $\gamma(t)=(t,\cosh(t))$, a partir do ponto $(0,1)$.
			\end{enumerate}
		
		\item [\sol{3}]...
		
		\linesep
		
		\item[\ex{4}] \blue{\textit{Mudanças de parâmetro:}}
		
		\begin{enumerate}[a.]
			\color{blue}
			\item Demonstrar que $s(\theta)=\dfrac{\theta^2}{\theta^2+1}$ é uma mudança de parâmetro diferenciável que transforma o invervalo $(0\infty)$ no intervalo $(0,1)$.
			
			\item Mostrar que a função $\lambda:(-1,1)\to(-\infty,+\infty)$ definida por $\lambda(t):=\tan(\pi t/2)$ é uma mudança de parâmetro.
			
			\item Provar que qualquer curva pode ser reparametrizada de forma tal que o domínio da reparametrização seja um intervalo de extremos $0$ e $1$.
		\end{enumerate}
	
	\item [\sol{4}] ...
	
	\linesep
	
	\item[\ex{5}]\blue{
	Provar que a curva

$$\gamma(t)=\left(2t,\dfrac{2}{1+t^2}\right)$$

com $t>0$ é regular e é uma reparametrização de

$$\alpha(t)=\left(\dfrac{2\cos t}{1+\sin t},1+\sin t\right),~~t\in(-\pi/2,\pi/2)$$
}

	\item [\sol{5}]...
	
	\linesep
	
	\item[\ex{6}] \blue{Seja $\alpha(t)=\left(\frac{1}{\sqrt3}\cos t+\frac{1}{\sqrt2}\sin t,\dfrac{1}{\sqrt3}\cos t-\dfrac{1}{\sqrt2}\sin t\right)$. Reparametrizar $\alpha$ pelo comprimento do arco.
	}
	\item[\sol{6}] ...
	
	\linesep
	
	\item [\ex{7}] \blue{Mostre que, se todas as retas tangentes a uma curva regular passam por um mesmo ponto $P\in\mathbb{R}^2$ , então seu traço está contido em uma reta.}
	
	\item[\sol{7}]...
	
	\linesep
	
	\item [\ex{8}] \blue{Mostre que, se todas as retas normais a uma curva regular passam por um mesmo ponto $P\in\mathbb{R}^2$, então seu traço está contido em um círculo.}
	
	\item[\sol{8}]
	\end{enumerate}
	
\newpage

% \addcontentsline{toc}{section}{Referências}
\bibliographystyle{plain}
\bibliography{refs}
\end{document}