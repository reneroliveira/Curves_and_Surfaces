\documentclass[12pt,letterpaper]{article}

\usepackage[brazilian]{babel}
\usepackage[utf8]{inputenc}
\usepackage[T1]{fontenc}

\usepackage{fullpage}
\usepackage{cancel}
\usepackage[top=2cm, bottom=4.5cm, left=2.5cm, right=2.5cm]{geometry}
\usepackage{amsmath,amsthm,amsfonts,amssymb,amscd}
\usepackage{lastpage}
\usepackage{enumerate}
\usepackage{fancyhdr}
\usepackage{mathrsfs}
\usepackage{xcolor}
\usepackage{graphicx}
\usepackage{listings}
\usepackage{hyperref}

\hypersetup{%
	colorlinks=true,
	linkcolor=blue,
	linkbordercolor={0 0 1}
}

\setlength{\parindent}{0.0in}
\setlength{\parskip}{0.05in}

% Edit these as appropriate
\newcommand\course{Rener Oliveira}
\newcommand\lcur{\mathcal{L}}
\newcommand{\linesep}{{\color{black} \rule{\linewidth}{0.5mm} }}
\newcommand{\rpos}{\mathbb{R}_{>0}}
\newcommand{\ex}[1]{\textcolor{blue}{\textbf{Exercício #1}}}
\newcommand{\sol}[1]{\textbf{Solução #1}}
\newcommand{\blue}[1]{{\color{blue}{#1}}}
\pagestyle{fancyplain}
\headheight 35pt        
\chead{\textbf{\Large Lista 3 \\ Curvas e Superfícies}}
\rhead{\small{\course \\ \today}}
\lfoot{}
\cfoot{}
\rfoot{\small\thepage}
\headsep 1.5em

\begin{document}
	
	\begin{enumerate}
		
		\item [\ex{1}] \textcolor{blue}{Verifique a regularidade e calcule o comprimento de arco e a curvatura das seguintes curvas, quando possível:}
		
		
		\begin{itemize}
			\blue{
			\item (retas) $\alpha(t)=(a+ct,b+dt),t\in\mathbb{R};$}
		
			Verificando regularidade:
			
			$\alpha'(t)=(c,d)$, que é diferente do vetor nulo para todo $t$, desde que $c$ e $d$ não sejam ambos nulos, pois neste caso não se trataria de um reta, mas sim de um ponto isolado. \textbf{Logo $\alpha$ é regular.}
		
			
			Calculando o comprimento de arco:
			
			\begin{align*}
				\displaystyle\lcur(t)&=\int_{t_0}^t||\alpha'(u)||du\\
				&=\int_{t_0}^t||(c,d)||du\\
				&=\int_{t_0}^t\sqrt{c^2+d^2}du\\
				&(t-t_0)\sqrt{c^2+d^2}
			\end{align*}
		
		Para a curvatura, vamos reparametrizar $\alpha$ por $\lcur(t)$ e calcular $\kappa(s)=\det(\alpha'(s),\alpha''(s))$ após a reparametrização.
		
		Para facilitar as contas, façamos $t_0=0$.
		
		\begin{align*}
			\alpha(\lcur(t))&=(a+ct\sqrt{c^2+d^2},b+dt\sqrt{c^2+d^2})\\
			\alpha'(s)&=(c\sqrt{c^2+d^2},d\sqrt{c^2+d^2})\\
			\alpha''(s)&=(0,0)\Rightarrow\\
			\kappa(s)&=\det(\alpha',\alpha'')=0
		\end{align*}
		
		\blue{
			\item $\alpha(t)=(t,t^4),t\in\mathbb{R};$}
		
		Verificando regularidade:
		
		$\alpha'(t)=(1,4t^3)\neq0$, pois a primeira componente é constante não-nula. \textbf{Logo a curva é regular.}
		
		Calculando o comprimeiro de arco:
		\begin{align*}
			\displaystyle\lcur(t)&=\int_{t_0}^t||\alpha'(u)||du\\
			&=\int_{t_0}^t||(1,4u^3)||du\\
			&=\int_{t_0}^{t}\sqrt{16u^6+1}du
		\end{align*}
	
		Seja $I=\int_{t_0}^{t}\sqrt{16u^6+1}du$, assim temos
		
		 $I'=\sqrt{16t^6+1}-\sqrt{16t_0^6+1}$ e
		 
		  $I''=\dfrac{48t^5}{\sqrt{16t^6+1}}$
		  
		 A reparametrização por comprimento de arco será $\alpha(s)=(I,I^4)$
		 A curvatura será:
		 
		 \begin{align*}
		 	\kappa(s)&=\det(\alpha'(s),\alpha''(s))\\
		 	&=\det\left[(I',4I^3I');(I'',4I^3I''+12I^2I'I')\right]\\
		 	&=I'(4I^3I''+12I^2I'^2)-4I^3I'I''\\
		 	&=\cancel{4I^3I'I''}+12I'(II'')^2-\cancel{4I^3I'I''}\\
		 	&=12I'(II'')^2
		 \end{align*}
	 
	 O resultado fica em termos da integral que não consegui calcular.		
	 
		\blue{
			\item (círculos) $\alpha(s)=(a+r\cdot\cos(s/r),b+r\cdot\sin(s/r)), s\in\mathbb{R},r>0;$}
		
		Verificando regularidade:
		
		$\alpha'(s)=(-r\cdot\sin(s/r)\cdot1/r,r\cdot\cos(s/r)\cdot1/r)\\=(-\sin(s/r),\cos(s/r))$
		
		Tal velocidade será sempre diferente do vetor nulo, pois os pontos na qual a primeira coorderada zera pertence ao conjunto $A=\{rk\pi;k\in\mathbb{Z}\}$, ou seja, os múltiplos de $0$ e $\pi$ radianos do círculo trigonométrico (ajustado por r);
		
		Já os pontos que a segunda componente zera pertencem à $B=\{r\pi/2+rk\pi;k\in\mathbb{Z}\}$.
		
		Se $s\in A$, $\alpha(s)=(0,1)$ para $k$ par e $\alpha(s)=(0,-1)$ para $k$ ímpar.
		
		Se $s\in B$, $\alpha(s)=(1,0)$ para $k$ par e $\alpha(s)=(-1,0)$ para $k$ ímpar.
		
		\textbf{Logo o círculo é regular}
		
		Calculando o comprimento de arco:
		
		\begin{align*}
			\lcur(t)&=\int_{t_0}^{t}||\alpha'(u)||du\\
			&=\int_{t_0}^{t}||(-\sin(u/r),\cos(u/r))||du\\
			&=\int_{t_0}^t(\sin^2(u/r)+\cos^2(u/r))^{1/2}du\\
			&=\int_{t_0}^tdu\\
			&=t-t_0
		\end{align*}
	
	Fazendo $t_0=0$ a fórmula original da curva já é a reparametrização por comprimento de arco.
	
	Calculando a curvatura:
	
	\begin{align*}
		\kappa(s)&=\det(\alpha'(s),\alpha''(s))\\
		&=\det\left[(-\sin(s/r),\cos(s/r));\frac1r(-\cos(s/r),-\sin(s/r))\right]\\
		&=\frac1r\left[\sin^2(s/r)+\cos^2(s/r)\right]\\
		&=\frac1r
	\end{align*}

	
	
		\blue{
			\item (cardióide) $\alpha(t)=(\cos(t)\cdot(2\cos(t)-1),\sin(t)\cdot(2\cos(t)-1)),t\in\mathbb{R};$}
		
		Verificando regularidade:
		
		\begin{align*}
			\alpha'(t)&=\left(-\sin(t)(2\cos(t)-1)-\cos(t)\cdot2\sin(t),\cos(t)(2\cos(t)-1)-\sin(t)\cdot2\sin(t)\right)\\
			&=(-4\sin(t)\cos(t)+\sin(t),2(\cos^2(t)-\sin^2(t))-\cos(t))\\
			&=(-2\sin(2t)+\sin(t),2\cos(2t)-\cos(t))
		\end{align*}
		
		Igualando a primeira componente à zero e resolvendo para $t$. temos:
		
		\begin{align*}
			&2\sin(2t)=\sin(t)\\
			&\Rightarrow4\sin(t)\cos(t)=\sin(t)\\
			&\Rightarrow\sin(t)=0\text{ ou }
			4\cos(t)=1	\\
			&\Rightarrow t=k\pi\text{ ou }t=2k\pi+\arccos(1/4)\text{ ou }t=2k\pi-\arccos(1/4)
		\end{align*}
	
	No caso $t=k\pi$, a segunda componente será:
	
	$$2\cos(2k\pi)-\cos(k\pi)=2-1=1\text{ para k par e}$$
	
	$$=2-(-1)=3\text{ para k ímpar}$$
	
	No caso $t=2k\pi+\arccos(1/4)$, por Pitágoras\footnote{ou Relação Fundamental da Trigonometria, o que preferir.} $\sin(t)=\sqrt{1-\frac1{16}}=\dfrac{\sqrt{15}}4$, assim, a segunda componente será:
	
	\begin{align*}
		&2(\cos^2(t)-\sin^2(t))-\cos(t)\\
		&=2\left(\dfrac1{16}-\dfrac{15}{16}\right)-\dfrac14\\
		&=2\dfrac{-14}{16}-\dfrac14\\&=-\dfrac{7}{4}-\dfrac14\\&=-2
	\end{align*}

	No último caso $t=2k\pi-\arccos(1/4)$, teremos $\sin(t)=-\dfrac{\sqrt{15}}4$ por simetria no círculo trigonométrico. Entretando o cálculo da segundo componente não muda, pois o único lugar que usa o seno, usa-o quadrático.
	
	Vemos então que os pontos que zeram a primeira coordenada não zeram a segunda.
	
	Resta verificar se os pontos que zeram a segunda componente também não zeram a segunda, se isso for provado, teremos uma curva regular.
	
	\textbf{Calculando o comprimento de arco}
	
	{\Huge{\textbf{\textit{COMPLETAR}}}}
	
		\blue{
			\item (catenária) $\alpha(t)=(t,\cosh(t)),t\in\mathbb{R}$
	}
		\end{itemize}
	
		\item[\ex{2}] \textcolor{blue}{Considere a elipse $\beta(t) =(a\cos(t),b\sin(t)),t\in\mathbb{R}$, onde $a >0,~b >0$ e $a\neq b$. Obtenhaos valores de $t$ onde a curvatura de $\beta$ é máxima e mínima.}
		
		\item[\sol{2}] É fácil ver que $\alpha$ é regular, por argumentos que já usamos anteriormente sobre pontos onde seno e cosseno zeram. Sendo assim, a curva admite reparametrização por comprimento de arco, fazendo o vetor tangente ter norma 1. Como achei muito difícil calcular a integral da parametrização, usarei o vetor  $T(t)=\dfrac{\alpha'(t)}{||\alpha'(t)||}$ que é velocidade unitária, e assim, pela definição de curvatura de \cite{pressley2001elementary} 
		
		
		incompleto por que estou perdido
		\begin{align*}
			\lcur(t)&=\int_{t_0}^t||(a\cos(u),b\sin(u))'||du\\
			&=\int_{t_0}^t||(-a\sin(u),b\cos(u))||du\\
			&=\int_{t_0}^t\left(a^2\sin(u)+b^2\cos^2(u)\right)^{\frac12}
		\end{align*}
		
		
		
		
		
		
	\end{enumerate}
	
	\newpage
	
	% \addcontentsline{toc}{section}{Referências}
	\bibliographystyle{plain}
	\bibliography{refs}
\end{document}