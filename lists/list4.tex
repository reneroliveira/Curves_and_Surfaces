\documentclass[12pt,letterpaper]{article}

\usepackage[brazilian]{babel}
\usepackage[utf8]{inputenc}
\usepackage[T1]{fontenc}

\usepackage{fullpage}
\usepackage{cancel}
\usepackage[top=2cm, bottom=4.5cm, left=2.5cm, right=2.5cm]{geometry}
\usepackage{amsmath,amsthm,amsfonts,amssymb,amscd}
\usepackage{lastpage}
\usepackage{enumerate}
\usepackage{fancyhdr}
\usepackage{mathrsfs}
\usepackage{xcolor}
\usepackage{graphicx}
\usepackage{listings}
\usepackage{hyperref}

\hypersetup{%
	colorlinks=true,
	linkcolor=blue,
	linkbordercolor={0 0 1}
}

\setlength{\parindent}{0.0in}
\setlength{\parskip}{0.05in}

% Edit these as appropriate
\newcommand\course{Rener Oliveira}
\newcommand\lcur{\mathcal{L}}
\newcommand{\real}{\mathbb{R}}
\newcommand{\rr}{\mathbb{R}^2}
\newcommand{\rn}{\mathbb{R}^n}
\newcommand{\linesep}{{\color{black} \rule{\linewidth}{0.5mm} }}
\newcommand{\rpos}{\mathbb{R}_{>0}}
\newcommand{\ex}[1]{\textcolor{blue}{\textbf{Exercício #1}}}
\newcommand{\sol}[1]{\textbf{Solução #1}}
\newcommand{\blue}[1]{{\color{blue}{#1}}}
\newcommand{\bd}[1]{\boldsymbol{#1}}
\pagestyle{fancyplain}
\headheight 35pt        
\chead{\textbf{\Large Lista 4 \\ Curvas e Superfícies}}
\rhead{\small{\course \\ \today}}
\lfoot{}
\cfoot{}
\rfoot{\small\thepage}
\headsep 1.5em

\begin{document}
	\begin{enumerate}
		
		\item [\ex{1}] \textcolor{blue}{Verifique se as seguintes curvas são 2-regulares:}
		\begin{enumerate}[(a)]
			\blue{
			\item $\alpha(t)=(t,t^2,t^3),t\in\real$
			}
			
		Veja que $\alpha'(t)=(1,2t,3t^2)$ e $\alpha''(t)=(0,2,6t)$ que é diferente do vetor nulo, para todo $t$ real por conta da componente central constante igual a $2$. Logo $\alpha$ é 2-regular.
		
		\blue{
			\item $\alpha(t)=(t,t^2+2,t^3+t),t\in\real$}
		
		\begin{align*}
			\alpha'(t)&=(1,2t,3t^2)\\
			\alpha''(t)&=(0,2,6t)\neq0\forall t\in\real,
		\end{align*}
	logo é 2-regular
	\end{enumerate}
	\item[\ex{3}] \textcolor{blue}{Obtenha uma reparametrização por comprimento de arco da curva }
	\blue{
	$$\alpha(t)=(e^t\cos(t),e^t\sin(t),e^t),~t\in\real$$}
	\item[\sol{3}] Encontrando a função comprimento de arco:
	
	\begin{align*}
		\lcur(t)&=\int_{t_0}^{t}||\alpha'(u)||du\\
		&=\int_{t_0}^{t}||e^u(-\sin u,\cos u,1)||du\\
		&=\int_{t_0}^{t}e^u(\sin^2u+\cos^2u+1)^{1/2}du\\
		&=\sqrt2\int_{t_0}^{t}e^udu\\
		&=\sqrt2(e^t-e^{t_0})
	\end{align*}
		
		Tomando os devidos cuidados com os domínios e imagens, podemos inverter a função comprimeiro de arco, fixando  gerando $\lcur^{-1}(t)=\ln\left(\dfrac{t}{\sqrt2}+e^{t_0}\right)$
		
		Assim, tomando $s_t=\dfrac{t}{\sqrt2}+e^{t_0}$ a reparametrização por comprimeiro de arco será:
		
		\begin{align*}
			\alpha(\lcur^{-1}(t))&=s_t\left(\cos\left(s_t\right),\sin\left(s_t\right),1\right),
		\end{align*}
	
	Onde está definida para valores de $t$ na qual $\dfrac{t}{\sqrt2}+e^{t_0}>0\Rightarrow t>-e^{t_0}\sqrt2$.
	
		
	\end{enumerate}
\end{document}