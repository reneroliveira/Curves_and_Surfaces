\documentclass[12pt,letterpaper]{article}

\usepackage[brazilian]{babel}
\usepackage[utf8]{inputenc}
\usepackage[T1]{fontenc}

\usepackage{fullpage}
\usepackage{cancel}
\usepackage[top=2cm, bottom=4.5cm, left=2.5cm, right=2.5cm]{geometry}
\usepackage{amsmath,amsthm,amsfonts,amssymb,amscd}
\usepackage{lastpage}
\usepackage{enumerate}
\usepackage{fancyhdr}
\usepackage{mathrsfs}
\usepackage{xcolor}
\usepackage{graphicx}
\usepackage{listings}
\usepackage{hyperref}
\usepackage{biblatex}
\bibliography{../lists/refs}
%\usepackage[backend=bibtex]{biblatex}
%\bibliography{../lists/refs}

\hypersetup{%
	colorlinks=true,
	linkcolor=blue,
	linkbordercolor={0 0 1}
}

\setlength{\parindent}{0.0in}
\setlength{\parskip}{0.05in}

% Edit these as appropriate
\newcommand\course{Rener Oliveira}
\newcommand\lcur{\mathcal{L}}
\newcommand{\real}{\mathbb{R}}
\newcommand{\rr}{\mathbb{R}^2}
\newcommand{\rn}{\mathbb{R}^n}
\newcommand{\linesep}{{\color{black} \rule{\linewidth}{0.5mm} }}
\newcommand{\rpos}{\mathbb{R}_{>0}}
\newcommand{\ex}[1]{\textcolor{blue}{\textbf{Exercício #1}}}
\newcommand{\sol}[1]{\textbf{Solução #1}}
\newcommand{\blue}[1]{{\color{blue}{#1}}}
\newcommand{\bd}[1]{\boldsymbol{#1}}
\pagestyle{fancyplain}
\headheight 35pt        
\chead{\textbf{\Large Teorema dos 4 Vértices}}
\lhead{Curvas e Superfícies\\
Trabalho A1}
\rhead{\small{\course \\ \today}}
\lfoot{}
\cfoot{}
\rfoot{\small\thepage}
\headsep 1.5em


\begin{document}
%	\tableofcontents
	\newpage
	\section{Curvatura com Sinal}
	
	Antes de prosseguirmos com o teorema, vamos definir curvatura com sinal.
	
	Dada um curva plana $\gamma:I\subset\real\to\rr$, com $I$ aberto, curva regular e \textit{unit-speed}, considere o vetor tangente (unitário), $T(s)=\dfrac{d\gamma}{ds}$.
	
	Seja $N$ o vetor normal à curva, unitário, obtido à partir de uma rotação de 90 graus no sentido horário de $T$.
	
	Pela Proposição 1.2.4 de \cite{pressley2001elementary}, ou pelo Exercício 7 da \href{https://github.com/reneroliveira/Curves_and_Surfaces/blob/main/lists/list1.pdf}{Lista 1}, o vetor $T'(s)=\dfrac{dT(s)}{ds}$ é ortogonal a $T(s)$.
	
	Desse forma $T'//N$, definimos então a curvatura com sinal, como sendo o múltiplo $\kappa_s$ tal que 
	
	$$T'=\kappa_sN$$
	
	No caso em que $\gamma$ não é \textit{unit-speed}, considere $J\subset\real$ um intervalo de reparametrização obtido pela função $h:I\to J$, via $h(t)=\int_{t_0}^t||\gamma'(u)||du$. Para simplificar a notação, considere $h(t)=s$.
	
	Seja $\overline{\gamma}(s)$
	
	\newpage
	\addcontentsline{toc}{section}{Referências}
	\printbibliography
\end{document}